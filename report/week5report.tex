\documentclass[11pt]{article}

\usepackage{times}
\usepackage[english]{babel}

% -----------------------------------------------
% especially use this for you code
% -----------------------------------------------

\usepackage{courier}
\usepackage{listings}
\usepackage{color}
\usepackage{tabularx}
\usepackage{graphicx}

\definecolor{Gray}{gray}{0.95}

\definecolor{mygreen}{rgb}{0,0.6,0}
\definecolor{mygray}{rgb}{0.5,0.5,0.5}
\definecolor{mymauve}{rgb}{0.58,0,0.82}

\lstset{language=C++,
	basicstyle = \normalsize\ttfamily,   % the size and fonts that are used
	tabsize = 2,                    % sets default tabsize
	breaklines = true,              % sets automatic line breaking
	keywordstyle=\color{blue}\ttfamily,
	stringstyle=\color{red}\ttfamily,
	commentstyle=\color{mygreen}\ttfamily,
	numbers=left,
	keepspaces=true,
	showspaces=false,
	showstringspaces=false,
}

\begin{document}

\title{Programming in C/C++ \\
       Exercises set five: grammatical parsers
}
\date{\today}
\author{Christiaan Steenkist \\
Jaime Betancor Valado \\
Remco Bos \\
}

\maketitle
\section*{Exercise 35, separated lists}
We fixed the list grammar so it does what we want it to.

\subsection*{Design}
To accomodate multiple datatypes we use listtokens instead of WORDs.
To allow empty lists, lists can exist out of nothing (empty OR).
Because a list with one item can't be distinguished as a separated list we just count it as a normal list.
A separated list is thus counted as \texttt{[listtoken ',' listtoken]} with optional repetition of \texttt{[listtoken ',']}.
We abbreviated these nonterminals as \texttt{sepstart} and \texttt{sepend}.
To simplify some more we defined \texttt{[listtoken ',']} as \texttt{septoken}.

\subsection*{Code listings}
\lstinputlisting[caption = grammar.gr]{src/a35/grammar.gr}

\end{document}